\documentclass{article}

\usepackage{graphicx}

\usepackage[utf8]{inputenc}
\usepackage[a4paper, scale=0.90]{geometry}
\usepackage{CJKutf8}

\usepackage{wrapfig}

\usepackage{float}

\usepackage{caption}
\usepackage{subcaption}

\usepackage{amsmath}
\usepackage{amssymb}

\usepackage{hyperref}

% Usage: \begin{Chinese}中文\end{Chinese}
\newenvironment{Chinese}
{\begin{CJK}{UTF8}{gbsn}}
{\end{CJK}}

\title{
    {2023 Spring Numerical Analysis Course Project}
}
\author{\begin{Chinese}程卓\end{Chinese} (2021011617)}

\begin{document}

\maketitle

\tableofcontents

\section{Introduction}
Code could be viewed on:

\url{https://git.tsinghua.edu.cn/chengz21/hello-bvp}

\url{https://github.com/zhuocheng2004/HelloBVP}

(equivalent)

\bigskip

NOTICE: many things are in the comments of the code.

\section{Problem}
Our goal is to solve the 2-point boundary problem:
$$
    u^{\prime\prime}(x) + p(x)u^\prime(x) + q(x)u(x) = f(x)
$$
on interval $[a, c]$
with homogenous boundary condition:
\begin{align}
    \zeta_{l0} \cdot u(a) + \zeta_{l1} \cdot u^\prime(a) = 0 \\
    \zeta_{r0} \cdot u(c) + \zeta_{r1} \cdot u^\prime(c) = 0
\end{align}


For any non-homogeneous boundary condition, 
the system can be expressed as the sum of two system, 
one has homogeneous boundary condition, 
the other has a linear (or quadratic) solution.

Our program implements the algorithm discussed in article 
{\sl A Fast Adaptive Numerical Method for Stiff Two-point Boundary Value Problems}
by June-Yub Lee and Leslie Greengard.

I do not follow the algorithm directly presented in the article. 
Instead, I implemented it based on my own understanding.

The resulting accuracy and speed are not very good, compared to that in the paper.
As I started working on this project late,
I have no time to investigate further.

\section{Programming Tasks}
\subsection{Discretization of integral equation}
See {\bf cheb.py}

\subsection{Full Algorithm Implementation}
I didn't follow exactly the algorithm written explicitly in the article.

I implemented it based on my understanding.

See {\bf cheb.py bvp.py btree.py solver.py}

\subsection{Newton's Iteration for nonlinear equation}
Not finished.

\subsection{Binary Tree Structure}
See {\bf btree.py}

\section{Theoretical Questions}
\subsection{Rank od matrix A}
I'm not sure, but it seems that if we choose too many Chebyshev points, 
the matrix will become near singular.

\subsection{Initial Guess of Newton's Method}

\subsection{Starr and Rokhlin}

\subsection{Generalization to PDE}

\section{Demos}

The following demos are run under environment:
\bigskip

Arch: amd64 (chip: AMD Ryzen 7 4800U with Radeon Graphics)

OS: Ubuntu/Linux 22.04 jammy

Python: 3.10.4

NumPy: 1.22.4

Matplotlib: 3.7.0

It seems that my program didn't behave well on the stiff problems in the article, 
and I still don't know why.

We used 64 sample points for the solution int demo\_1 and demo\_2 

\subsection{demo\_sigma}

This demo computes the function $\sigma$ in two ways: 

1. use the whole interval.

2. cut the interval into $2^6 = 64$ sub-intervals using a 6-level binary tree.

Note: in whole interval case, we use 32 Chebyshev points. 
If we use more points, we might encounter `singular matrix' error when computing.

\begin{figure}[H]
	\centering
    \begin{subfigure}[h]{0.8\linewidth}
	    \includegraphics[width=\textwidth]{demo_sigma.png}
    \end{subfigure}
\end{figure}

\subsection{demo\_1}

Problem to solve:
$$
    u^{\prime\prime}(x) = 1
$$ on $[0, 1]$
with boundary condition:
$$
    u(0) = u^\prime(1) = 0
$$

Actual solution:
$$
    u(x) = \frac12 x^2 - x
$$

\begin{figure}[H]
	\centering
	\begin{subfigure}[h]{0.45\linewidth}
	    \includegraphics[width=\textwidth]{demo_1_1.png}
    \end{subfigure}
    \begin{subfigure}[h]{0.45\linewidth}
	    \includegraphics[width=\textwidth]{demo_1_2.png}
    \end{subfigure}
\end{figure}

Running time: 13.509s

Computing process ($L^\infty$ tolerance = $10^{-8}$):

NOTE: the `relative error' is the estimated relative error during 
computing process, not the relative error compared with the real solution.

\begin{table}[H]
    \centering
    \begin{tabular}{l|r|r}
        step & number of sub-intervals & $L^\infty$ relative error \\
        \hline
        1  &   1 & 0.010929542363862177 \\
        2  &   2 & 0.002732385590965558 \\
        3  &   4 & 0.0006830963977413618 \\
        4  &   8 & 0.00017077409943544453 \\
        5  &  16 & 4.2693524858777865e-05 \\
        6  &  32 & 1.0673381214965083e-05 \\
        7  &  64 & 2.668345304712716e-06 \\
        8  & 128 & 6.670863279545358e-07 \\
        9  & 256 & 1.6677158029554384e-07 \\
        10 & 512 & 4.1692901464607246e-08 \\
    \end{tabular}
    \caption{Refinement Process}
\end{table}

%%%%

\subsection{demo\_2}

Problem to solve:
$$
    u^{\prime\prime}(x) + u(x) = 1
$$ on $[0, 6\pi]$
with boundary condition:
$$
    u(0) = u^\prime(6\pi) = 0
$$

Actual solution:
$$
    u(x) = 1 - \cos(x)
$$

\begin{figure}[H]
	\centering
	\begin{subfigure}[h]{0.45\linewidth}
	    \includegraphics[width=\textwidth]{demo_2_1.png}
    \end{subfigure}
    \begin{subfigure}[h]{0.45\linewidth}
	    \includegraphics[width=\textwidth]{demo_2_2.png}
    \end{subfigure}
\end{figure}

Running time: 18.520s

Computing process ($L^\infty$ tolerance = $10^{-5}$):

\begin{table}[H]
    \centering
    \begin{tabular}{l|r|r}
        step & number of sub-intervals & $L^\infty$ relative error \\
        \hline
        1  &   1 & 0.812113156745877 \\
        2  &   2 & 0.5064112496456088 \\
        3  &   4 & 0.610392895761098 \\
        4  &   7 & 0.47235297036986823 \\
        5  &  14 & 0.012047083860414618 \\
        6  &  28 & 0.0029389703437309564 \\
        7  &  56 & 0.0005390607459993711 \\
        8  & 109 & 0.00012682070475158503 \\
        9  & 215 & 5.151907746078857e-05 \\
        10 & 426 & 3.055720983986894e-05 \\
        11 & 845 & 2.823770749495437e-06 \\
    \end{tabular}
    \caption{Refinement Process}
\end{table}

You can see that it's indeed somewhat adaptive, although apparently something seemed to go wrong
(we shouldn't have so much subdivision).

\end{document}
